\section{Best Practises}

\subsection{Git}

\subsubsection{Anlegen eines neuen lokalen Repositories}

\begin{itemize}
	\item mkdir $<$Verzeichnisname$>$
	\item cd $<$Verzeichnisname$>$
	\item git init
\end{itemize}

\subsubsection{Anlegen eines neuen Repositories auf dem Server}

\begin{itemize}
	\item mkdir $<$Verzeichnisname$>$
	\item cd $<$Verzeichnisname$>$
	\item git init --bare
\end{itemize}

Beispiel:

\begin{itemize}
	\item mkdir cg
	\item cd cg
	\item git init --bare
\end{itemize}

\subsubsection{Setzen eines Remote Repositories}

\begin{itemize}
	\item git remote add $<$Name des Remote Repositories$>$ $<$Pfad zum Repository$>$
	\item Beispiel (gleicher Rechner): git remote add local /Users/abo781/repository/cg
	\item Beispiel (Server): git remote add server ssh://abo781@git.informatik.haw-hamburg.de/srv/git/computergrafik/cg
\end{itemize}

\subsubsection{Lokale �nderungen an Server Repository senden}

\begin{itemize}
	\item git push $<$Name des Remote Repositories$>$ $<$Name des Branches$>$
	\item Beispiel: git push server master
\end{itemize}

\subsubsection{Lokales Klon-Repository von Server holen}
\label{section:clone}

\begin{itemize}
	\item git clone $<$Name des Remote Repositories$>$ 
	\item Beispiel: git clone ssh://abo781@git.informatik.haw-hamburg.de/srv/git/computergrafik/cg
\end{itemize}

\subsection{Sonar}

Starten von Sonar mit
\begin{verbatim}
cd <Sonar-Verzeichnis>/bin/<Betriebssystem>/
./sonar.sh start
\end{verbatim}

Pr�fen, ob der Sonar-Server l�uft (Im Browser):
\begin{verbatim}
http://localhost:9000/
\end{verbatim}

Sonar-Analyse durchf�hren (ben�tigt Maven-Installation).
\begin{verbatim}
mvn sonar:sonar
\end{verbatim}